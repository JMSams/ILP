\documentclass[a4paper, 12pt]{article}

\usepackage[utf8]{inputenc}

\usepackage[margin=1in]{geometry}

\usepackage[parfill]{parskip}

\usepackage{lastpage}

\usepackage{float}

\usepackage{titling}

\usepackage{fancyhdr}
\pagestyle{fancy}
\renewcommand{\headrulewidth}{0pt}
\lhead{}\chead{}\rhead{}
\lfoot{}\cfoot{}\rfoot{}

\title{ILP Proposal}
\author{Jamie M Sams (S153907)}
\date{\today}

\usepackage[british]{babel}

\usepackage[style=authoryear, backend=bibtex, dashed=false]{biblatex}
\addbibresource{../ILP-References.bib}

\usepackage[hidelinks]{hyperref}

\begin{document}

\pagenumbering{gobble}

\begin{titlepage}
	\vspace*{\fill}
	\centering
	{\huge \thetitle}\\[.5cm]
	{\large \theauthor}\\[.5cm]
	{MSc Game Development}
	\vspace*{\fill}
\end{titlepage}

\pagenumbering{arabic}
\lhead{\theauthor}
\rhead{MSc Game Development}
\lfoot{Proposal - ILP}
\rfoot{Page \thepage{}/\pageref*{LastPage}}

For my Individual Learning Plan I will explore how procedural algorithms can be used to create mazes and dungeons for use in both 2D and 3D games.

As a starting point for this project, I have researched and implemented several maze algorithms in Unity.  These include the Binary Tree, Recursive Backtracker and Kruskal algorithms, detailed by \cite{Buck2015}.  So far the implementaion is a simple 2D view of the maze as it's being generated, using Unity's coroutines to show the proccess step by step.



\pagebreak
\section*{Bibliography}

\printbibliography[heading=none]

\end{document}