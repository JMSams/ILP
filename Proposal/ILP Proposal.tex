\documentclass[a4paper, 12pt]{article}

\usepackage[utf8]{inputenc}

\usepackage[margin=1in]{geometry}

\usepackage{lastpage}

\usepackage{float}

\usepackage{titling}

\usepackage{fancyhdr}
\pagestyle{fancy}
\renewcommand{\headrulewidth}{0pt}
\lhead{}\chead{}\rhead{}
\lfoot{}\cfoot{}\rfoot{}

\title{ILP Proposal}
\author{Jamie M Sams (S153907)}
\date{\today}

\usepackage[hidelinks]{hyperref}

\usepackage[british]{babel}

\usepackage[style=authoryear, backend=bibtex, dashed=false]{biblatex}
\addbibresource{../ILP-References.bib}

\begin{document}

\pagenumbering{gobble}

\begin{titlepage}
	\vspace*{\fill}
	\centering
	{\huge \thetitle}\\[.5cm]
	{\large \theauthor}\\[.5cm]
	{MSc Game Development}
	\vspace*{\fill}
\end{titlepage}

\pagenumbering{arabic}
\lhead{\theauthor}
\rhead{MSc Game Development}
\lfoot{Proposal - ILP}
\rfoot{Page \thepage{}/\pageref*{LastPage}}

For my Individual Learning Plan I will explore how procedural algorithms can be used to create mazes and dungeons for use in both 2D and 3D games.

As a starting point for this project, I have researched and implemented several maze algorithms in Unity.  These include the Binary Tree, Recursive Backtracker and Kruskal algorithms, detailed by \cite{Buck2015}.  So far the implementaion is a simple 2D view of the maze as it's being generated.

Taking this a step furthur, I propose to implement these algorithms into a game setting, using simple 2D assets to create a simple dungeon crawler type game that is different on every play through.

One step further would be the introduction of additional depth and complexity to the levels generated.  This would consist of layered mazes, possibly of increasing or decreasing size, that are linked by ladders, trapdoors or some other mechanism.

Going further, I will then expand into the third dimension, adding a first-person perspective to the generated mazes.

\pagebreak
\section*{Bibliography}

\printbibliography[heading=none]

\end{document}